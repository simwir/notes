\subsection{Method}
\begin{description}
    \item[Object] An entity with identity, state and behavior.
    \item[Class] A description of a collection of objects sharing structure, behavioral pattern, and attributes.
    \item[Problem domain] That part of a context that is administrated, monitored, or controlled by a system.
    \item[Application domain] The organization that administrates, monitors, or controls a problem domain.
    \item[System] A collection of components that implements modeling requirements, functions, and interfaces.
\end{description}

\subsection{System Choice}
\begin{description}
    \item[System definition] A concise description of a computerized system expressed in natural language.
\end{description}

\subsection{Problem-Domain Analysis}
\begin{description}
    \item[Problem domain] That part of a context that is administrated, monitored, or controlled by a system.
    \item[Model] A description of classes, objects, structures, and behavior in a problem domain.
\end{description}

\subsection{Classes}
\begin{description}
    \item[Object] An entity with identity, state, and behavior.
    \item[Class] A description of a collection of objects sharing structure, behavioral pattern, and attributes.
    \item[Event] An instantaneous incident involving one or more objects.
\end{description}

\subsection{Structures}
\large\textbf{\textit{Class structures}}\normalsize
\begin{description}
    \item[Generalization] A general class (the super class) describes properties common to a group of specialized classes (the subclasses).
    \item[Cluster] A collection of related classes.
\end{description}
\large\textbf{\textit{Object structures}}\normalsize
\begin{description}
    \item[Aggregation] A superior object (the whole) consists of a number of objects (the parts).
    \item[Association] A meaninful relation between a number of objects.
\end{description}

\subsection{Behavior}
\begin{description}
    \item[Event trace] A sequence of events involving a specific object.
    \item[Behavioral pattern] A description of possible event traces for all objects in a class.
\end{description}

\subsection{Application-Domain Analysis}
\begin{description}
    \item[Application domain] An organization that administrates, monitors, or controls a problem domain.
    \item[Requirements] A system's externally observable behavior.
\end{description}

\subsection{Usage}
\begin{description}
    \item[Actor] An abstraction of users or other systems that interact with the target system.
    \item[Use case] A pattern for interaction between the system and actors in the application domain.
\end{description}

\subsection{Functions}
\begin{description}
    \item[Function] A facility for making a model useful for actors.
\end{description}

\subsection{Interfaces}
\begin{description}
    \item[Interface] Facilities that make a system's model and functions available to actors.
    \item[User interface] An interface to users.
    \item[System interface] An interface to other systems.
\end{description}

\subsection{Criteria}
\begin{description}
    \item[Criterion] A preferred property of an architecture.
    \item[Conditions] The technical, organizational, and human opportunities and limits involved in performing a task.
\end{description}

\subsection{Components}
\begin{description}
    \item[Component architechture] A system structure of interconnected components.
    \item[Component] A collection of program parts that constitutes a whole and has well-defined responsibilities.
\end{description}

\subsection{Processes}
\begin{description}
    \item[Process architecture] A system-execution structure.
    \item[Processor] A piece of equipment that can execute a program.
    \item[Program component] A physical module of program code.
    \item[Active objects] An object that has been assigned a process.
\end{description}

\subsection{Component Design}
\begin{description}
    \item[Component] A collection of program parts that constitutes a whole and has well-defined responsibilities.
    \item[Connection] The implementation of a dependency relation.
\end{description}

\subsection{Model Component}
\begin{description}
    \item[Model component] A part of a system that implements the problem-domain model.
    \item[Attribute] A descriptive property of a class or an event.
\end{description}

\subsection{Function Component}
\begin{description}
    \item[Function component] A part of a system that implements functional requirements.
    \item[Operation] A process property specified in a cladd and activated through the class' objects.
\end{description}

\subsection{Connecting Component}
\begin{description}
    \item[Coupling] A measure of how closely two classes or components are connected.
    \item[Cohesion] A measure of how well a class or component is tied together.
\end{description}

