\begin{itemize}
	\item \textbf{Physically}, people interact by
	\begin{itemize}
		\item Pressing buttons
		\item Touching a screen
		\item Moving a mouse over a table
		\item Etc...
	\end{itemize}
	\item \textbf{Perceptually}, people interact through
	\begin{itemize}
		\item What they can see
		\item What they can hear
		\item What they can feel
	\end{itemize}
	\item \textbf{Conceptually}, people need to
	\begin{itemize}
		\item Know what commands exist
		\item Know that certain data is available
		\item Know how to navigate
		\item Be able to find details
		\item Be able to gain an overview 
	\end{itemize}
\end{itemize}
\textbf{WIMP}\\
\begin{itemize}
	\item Windows
	\item Icons
	\begin{itemize}
		\item Understandable
		\item Familiar
		\item Unambiguous
		\item Memorable
		\item Informative
		\item Few
		\item Distinct
		\item Attractive
		\item Legible
		\item Compact
		\item Coherent
		\item Extensible
	\end{itemize}
	\item Menus
	\begin{itemize}
		\item Cascading
		\item Pop-up
		\item Contextual
	\end{itemize}
	\item Pointers
\end{itemize}
\textbf{Nielsens heuristics (revised)}\\
\begin{enumerate}
	\item Visibility of system status
	\item Match between system and real world
	\item User control and freedom
	\item Consistency and standards
	\item Error prevention
	\item Recognition rather than recall
	\item Flexibility and efficiency of use
\end{enumerate}
\textbf{Petrie and Powers' heuristics}
\begin{itemize}
	\item Physical presentation
	\begin{enumerate}
		\item Make text and interactive elements large and clear enough
		\begin{itemize}
			\item Default and typically rendered sizes of text and interactive elements should be large enough to be easy to read and manipulate.
		\end{itemize}
		\item Make page layout clear
		\begin{itemize}
			\item Make sure that the layout of information on the page is clear, easy to read and reflects the organization of the material.
		\end{itemize}
		\item Avoid short time-outs and display times
		\begin{itemize}
			\item Provide time-outs that are long enough for users to complete the task comfortably, and if information is displayed for a limited time, make sure it is long enough for	users to read comfortably.
		\end{itemize}
		\item Make key content and elements and changes to them salient
		\begin{itemize}
			\item Make sure the key content and interactive elements are clearly visible on the page	and that changes to the page are clearly indicated.
		\end{itemize} 
	\end{enumerate}
	\item Content
	\begin{enumerate}[resume]
		\item Provide relevant and appropriate content
		\begin{itemize}
			\item Ensure that content is relevant to users' task and that it is appropriately and respectfully worded.
		\end{itemize}
		\item Provide sufficient but not excessive content
		\begin{itemize}
			\item Provide sufficient content (including Help) so that user can complete their task but	not excessive amounts of content that they are overwhelmed.
		\end{itemize}
		\item Provide clear terms, abbreviations, avoid jargon
		\begin{itemize}
			\item Define all complex terms, jargon and explain abbreviations.
		\end{itemize}
	\end{enumerate}
	\item Information Architecture
	\begin{enumerate}[resume]
		\item Provide clear, well-organized information structures
		\begin{itemize}
			\item Provide clear information structures that organize the content on the page and help users complete their task.
		\end{itemize}
	\end{enumerate}
	\item Interactivity
	\begin{enumerate}[resume]
		\item How and why
		\begin{itemize}
			\item Provide users with clear explanations of how the interactivity works and why things are happening.

		\end{itemize}
		\item Clear labels and instructions
		\begin{itemize}
			\item Provide clear labels and instructions for all interactive elements. Follow web conventions for labels and instructions (e.g. use of asterisk for mandatory elements).
		\end{itemize}
		\item Avoid duplication/excessive effort by users
		\begin{itemize}
			\item Do not ask users to provide the same information more than once and do not ask for
			excessive effort when this could be achieved more efficiently by the system.
		\end{itemize}
		\item Make input formats clear and easy
		\begin{itemize}
			\item Make clear in advance what format of information is required from users. Use input formats that are easy for users, such as words for months rather than numbers.
		\end{itemize}
		\item Provide feedback on user actions and system progress
		\begin{itemize}
			\item Provide feedback to users on their actions and if a system process will take time, on its progress.
		\end{itemize}
		\item Make the sequence of interaction logical
		\begin{itemize}
			\item Make the sequence of interaction logical for users (e.g. users who are native speakers
			of European languages typically work down a page from top left to bottom right, so provide the Next button at the bottom right).
		\end{itemize}
		\item Provide a logical and complete set of options
		\begin{itemize}
			\item Ensure that any set of options includes all the options users might need and that the set of options will be logical to users.
		\end{itemize}
		\item Follow conventions for interaction 
		\begin{itemize}
			\item Unless there is a very particular reason not to, follow web and logical conventions in the interaction (e.g. follow a logical tab order between interactive elements).
		\end{itemize}
		\item Provide the interactive functionality users will need and expect
		\begin{itemize}
			\item Provide all the interactive functionality that users will need to complete their task and that they would expect in the situation (e.g. is a search needed or provided?).
		\end{itemize}
		\item Indicate if links go to an external site or to another webpage
		\begin{itemize}
			\item If a link goes to another website or opens a different type of resource (e.g. PDF document) indicate this in advance.
		\end{itemize}
		\item Interactive and non-interactive elements should be clearly distinguished
		\begin{itemize}
			\item Elements which are interactive should be clearly indicated as such, and element which are not interactive should not look interactive.
		\end{itemize}
		\item Group interactive elements clearly and logically
		\begin{itemize}
			\item Group interactive elements and the labels and text associated with them in ways that make their functions clear.
		\end{itemize}
		\item Provide informative error messages and error recovery
		\begin{itemize}
			\item Provide error messages that explain the problem in the users' language and ways to recover from errors.
		\end{itemize}
	\end{enumerate}
\end{itemize}
\textbf{Depth perception}
\begin{itemize}
	\item Primary cues
	\begin{itemize}
		\item Retinal disparity (two separate images)
		\item Stereopsis (image combining process)
		\item Accommodation (muscular process to create image focus)
		\item Convergence (muscular process for image focus on short distances)
	\end{itemize}
	\item Secondary cues
	\begin{itemize}
		\item light and shade
		\item Linear perspective
		\item Height in horizontal plane
		\item Motion parallax
		\item Overlap
		\item Relative size
		\item texture gradient
	\end{itemize}
\end{itemize}
\textbf{Pattern recognition}\\
\textit{Gestalt laws of perception}
\begin{itemize}
	\item Proximity
	\begin{itemize}
		\item This law states that objects that are located close to one another will be perceived as being associated with one another, i.e. as belonging to a group, or as parts of a larger whole.
	\end{itemize}
	\item Continuity
	\begin{itemize}
		\item When a line (or objects arranged in a way that indicates a line) is perceived, which appears to have one or more branches, the branch which follows the direction of the original line most faithfully is perceived as being the continuation of the original path, and others are perceived as appendages.
	\end{itemize}
	\item Similarity
	\begin{itemize}
		\item Objects which share similar properties are assumed to have association with one another. These similar properties may be, for example, visual properties such as shape or colour. 
	\end{itemize}
	\item Closure
	\begin{itemize}
		\item Objects that are close together are perceived as being part of a whole, to the extent that gaps between them may be imagined to be 'closed', forming complete shapes or borders. Wertheimer notes than in many cases this is not the dominant factor --- Others may predominate. 
	\end{itemize}
	
\end{itemize}